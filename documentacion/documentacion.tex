% [Tamaño principal de la fuente del documento, tamaño del papel, con título (crea salto de página)]{Tipo de documento}
\documentclass[12pt, a4paper, titlepage]{article}
\usepackage[utf8]{inputenc}
% Traduce expresiones al español
\usepackage[spanish]{babel}
\usepackage{setspace}

\usepackage{graphicx}
\usepackage{geometry}
\geometry{margin=2.5cm}
\graphicspath{ {./capturas/} }

% Permite utilizar labeling para listar de forma personalizada
\usepackage{scrextend}
% Permite centrar verticalmente m{}
\usepackage{array}
% Permite colorear texto
\usepackage{xcolor}

\title{\LARGE \textbf{Proyecto Aplicación Restaurante} \\[2ex] \Large Android - Express - JavaFX}
\author{\\[20ex]Cristian Fernández}
\date{\today}

\begin{document}
\maketitle

\doublespacing
\noindent \\
\tableofcontents % Índice
\newpage


\section{Descripción del proyecto}
\noindent \\Este proyecto trata sobre la creación de una aplicación móvil, una aplicación de escritorio y una API-Rest \\

\newpage

\section{Problemas encontrados}
\noindent \\Falta de motivación. 
\newpage

\section{Bibliografía}
\noindent \\Enlaces a información consultada:
\begin{itemize}
    \item youtube
    \item stack overflow
    \item geeksforgeeks
    \item mongodb.com/community
    \item IA de google
\end{itemize}

\end{document}